% !TEX root = thesis.tex

%%%%%%%%%%%%%%%%%%%%%%
\chapter{Related Work}
%%%%%%%%%%%%%%%%%%%%%%

\TODO{Maybe move this up, toward the background section or introduction}

- Degrave paper https://arxiv.org/pdf/1412.4039
	Similar to what we're doing, but they don't consider arbitrary splits of delegations, 
	Propose calculating the final power through systems of linear equations

- Bersetche paper "A Voting Power Measure for Liquid Democracy with Multiple Delegation" \cite{bersetcheGeneralizingLiquidDemocracy2022}
	Here, they propose fractional delegation (called Multiple Delegation)
	Delegates can also retain power for themselves
	Also they propose a penalty factor, in order to XXX

- The Bertsche paper cites some practical experiments with LD, I could include those as well


- The viscous democracy paper might also be relevant

- I remember reading some papers that mention why fractional delegation is benefitial, if it was not the Bersetche paper I'll try to find it again..

% mention here the benefits of fractional delegation