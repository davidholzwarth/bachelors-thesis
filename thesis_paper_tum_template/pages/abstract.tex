\chapter{\abstractname}

This thesis explores fractional delegation in Liquid Democracy, where voters can split their vote among multiple delegates, aiming to reduce vote concentration and improve representational fairness. We formalize the mode and a method of resolving the final voting power of each participant. We then present and evaluate three implementations of this method: using a solver for systems of linear equations, a linear programming solver, and an iterative implementation. We prove that, under well-formed conditions, delegation graphs have a unique, power-conserving solution. A preprocessing pipeline ensures graphs are resolvable by eliminating problematic cycles. Through evaluation on synthetic, social, and real-world graphs, we find that the linear systems solver is fastest in most cases, while the iterative method struggles with cycles that retain power. Our findings demonstrate that fractional delegation is both feasible and scalable, paving the way for more expressive digital democratic systems.
