% !TEX root = ../main.tex

%%%%%%%%%%%%%%%%%%%%
\chapter{Conclusion and Further Research}
\label{chap:conclusion}
%%%%%%%%%%%%%%%%%%%%

This thesis explored the resolution of delegation graphs in Liquid Democracy under the extension of fractional delegation. By allowing voters to distribute their vote across multiple delegates, we addressed shortcomings of traditional Liquid Democracy, including vote concentration and vulnerability to closed delegation cycles.

We proposed a formal model of fractional delegation and introduced three implementations to compute final voting power: using a solver for systems of linear equations, using a linear programming solver, and an iterative simulation of power flow. We demonstrated that the linear systems formulation yields a unique solution and ensures conservation of power under well-formed graphs. Furthermore, we introduced a preprocessing pipeline that transforms arbitrary graphs into well-formed delegation graphs, enabling resolution of delegations even in the presence of complex structures like cycles.

Our evaluation across synthetic, social, and real-world graphs shows that the linear systems solver outperforms other methods on sparse graphs, while the iterative solver scales more favorably in dense settings, void of cycles which retain a lot of power. The LP-based method proved less efficient in most circumstances. 

Looking ahead, this thesis leaves several questions open. While we hypothesize that fractional delegation may reduce vote concentration by allowing voters to distribute their trust among multiple delegates, this claim requires empirical validation. A practical implemented fractional delegation platform would enable real-world testing of not only the resolution of delegations, but the entire Liquid Democracy process, including the collection of votes, and the calculation of which option wins the vote. Moreover, a user study could provide insight into how people delegate, and how easy it is for them to grasp and engage with fractional delegation, which is an open and critical question when it comes to evaluating how feasible, effective, and democratic Liquid Democracy with fractional delegation is. 
