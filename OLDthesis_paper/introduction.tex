% !TEX root = thesis.tex

%%%%%%%%%%%%%%%%%%%%%%
\chapter{Introduction}
%%%%%%%%%%%%%%%%%%%%%%

%In the classical democracies found in most western countries, each voter casts one vote for the candidate they wish to vote for. It is expected of voters to make an informed choice, based on the information available to them.  [cite fords paper or dahls "on democracy"] 

The introduction is a longer writeup that gently eases the reader into your
thesis~\cite{dinesh20oakland}. Use the first paragraph to discuss the setting.

In the second paragraph you can introduce the main challenge that you see.

The third paragraph lists why related work is insufficient.

The fourth and fifth paragraphs discuss your approach and why it is needed.

The sixth paragraph will introduce your thesis statement. Think how you can
distill the essence of your thesis into a single sentence.

The seventh paragraph will highlight some of your results

The eights paragraph discusses your core contribution.

This section is usually 3-5 pages.

Also include a description of how the paper is set up.